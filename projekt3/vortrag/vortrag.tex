\documentclass{beamer}


\usepackage{color}
\usepackage{listings}
\usepackage{courier}
\lstset{
basicstyle=\tiny\ttfamily, % Standardschrift
numbers=left, % Ort der Zeilennummern
tabsize=4, % Groesse von Tabs
}
\lstloadlanguages{C++}
%\DeclareCaptionFont{blue}{\color{blue}}
 
%\captionsetup[lstlisting]{singlelinecheck=false, labelfont={blue}, textfont={blue}}
\usepackage{caption}
\DeclareCaptionFont{white}{\color{white}}
\DeclareCaptionFormat{listing}{\colorbox{8}{\parbox{\textwidth}{\hspace{15pt}#1#2#3}}}
\captionsetup[lstlisting]{format=listing,labelfont=white,textfont=white, singlelinecheck=false, margin=0pt, font={bf,footnotesize}}

\usepackage[utf8x]{inputenc}
\usepackage{ngerman}
\usepackage{graphicx}




\title{Kleine Einfache Gruppen}
\author{Patrick Dabbert, Stephan Hilb und Martin R\"osner}
\date{\today}

\usepackage{beamerthemesplit}


\begin{document}

\begin{frame}
	\titlepage
\end{frame}

%\begin{frame}
%	\frametitle{Inhaltsverzeichnis}
%	\tableofcontents
%\end{frame}

\section{Aufgabe 1}

\begin{frame}
	\frametitle{Die Aufgabenstellung}
	\begin{quote}
	Aufgabe 1:\\
	Fertigen Sie eine Liste der endlichen einfachen nicht-abelschen Gruppen bis Ordnung 10.000 an.
	Sie dürfen hierbei verwenden, dass diese in folgender Liste enthalten sind: $A_n, PSL(2,q), PSL(3,3),
	PSU(3,3)$ und $M_{11}$. Verwenden sie, dass die Ordnung eines endlichen Körpers eine Primzahlpotenz ist.
	Welche der von ihnen gefundenen Gruppen sind minimal einfach?
		 
	\end{quote}
\end{frame}



\begin{frame}
	\frametitle{Definition:Einfache Gruppen}
	\begin{quote}
	 einfache Gruppen Definition
		 
	\end{quote}
\end{frame}
\begin{frame}
	\frametitle{Definition Minimal Einfache Gruppen}
	\begin{quote}
	 minimal einfache Gruppen Definition
		 
	\end{quote}
\end{frame}

\begin{frame}
	\frametitle{Lösungsschritte}
	\begin{quote}
	 einfache Gruppen Definition
		 
	\end{quote}
\end{frame}

\begin{frame}
	\frametitle{Implementierung in GAP}
	\begin{quote}
	 einfache Gruppen Definition
		 
	\end{quote}
\end{frame}

\begin{frame}
	\frametitle{Ergebnis}
	\begin{quote}
	 Tabelle ergebnis
		 
	\end{quote}
\end{frame}



\section{Aufgabe 2}

\begin{frame}
	\frametitle{Die Aufgabenstellung}
	\begin{quote}
	Aufgabe 2:\\
	Zeigen Sie, dass $A_8$ und $ PSL(3,4)$ nicht isomorph sind.
		 
	\end{quote}
\end{frame}

\begin{frame}
	\frametitle{Lösungsschritte}
	\begin{quote}
	 einfache Gruppen Definition
		 
	\end{quote}
\end{frame}

\begin{frame}
	\frametitle{Implementierung in GAP}
	\begin{quote}
	 einfache Gruppen Definition
		 
	\end{quote}
\end{frame}

\begin{frame}
	\frametitle{Ergebnis}
	\begin{quote}
	 Tabelle ergebnis
		 
	\end{quote}
\end{frame}


\section{Aufgabe 3}

\begin{frame}
	\frametitle{Die Aufgabenstellung}
	\begin{quote}
	Aufgabe3:\\
	(In Zusammenarbeit mit Partitionen-Gruppe)\\
	Bestimmen Sie, welche der von ihnen bestimmten Gruppen eine nicht-triviale Partition von Untergruppen besitzt.

	\end{quote}
\end{frame}

\begin{frame}
	\frametitle{Lösungsschritte}
	\begin{quote}
	 einfache Gruppen Definition
		 
	\end{quote}
\end{frame}

\begin{frame}
	\frametitle{Implementierung in GAP}
	\begin{quote}
	 einfache Gruppen Definition
		 
	\end{quote}
\end{frame}

\begin{frame}
	\frametitle{Ergebnis}
	\begin{quote}
	 Tabelle ergebnis
		 
	\end{quote}
\end{frame}

\section{Aufgabe 4}

\begin{frame}
	\frametitle{Die Aufgabenstellung}
	\begin{quote}
	Aufgabe 4:\\
	(In Zusammenarbeit mit Sylowgruppen-Gruppe) Bestimmen Sie, welche der von Ihnen gefundenen Gruppen N-Gruppen sind. Geben Sie außerdem für 
	$ P \in Syl_2(G) $ und $ P \in Syl_3(G) $ die Isomorphietypen von P, Z(P), $ C_G(P) $ und $ N_G(P) $ an, wobei G eine Gruppe aus ihrer Liste ist.
	\end{quote}
\end{frame}

\begin{frame}
	\frametitle{Lösungsschritte}
	\begin{quote}
	 einfache Gruppen Definition
		 
	\end{quote}
\end{frame}

\begin{frame}
	\frametitle{Implementierung in GAP}
	\begin{quote}
	 einfache Gruppen Definition
		 
	\end{quote}
\end{frame}

\begin{frame}
	\frametitle{Ergebnis}
	\begin{quote}
	 Tabelle ergebnis
		 
	\end{quote}
\end{frame}

\end{document}
