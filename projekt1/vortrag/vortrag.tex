\documentclass{beamer}


\usepackage{color}
\usepackage{listings}
\usepackage{courier}
\lstset{
basicstyle=\footnotesize\ttfamily, % Standardschrift
numbers=left, % Ort der Zeilennummern
tabsize=4, % Groesse von Tabs
}
\lstloadlanguages{C++}
%\DeclareCaptionFont{blue}{\color{blue}}
 
%\captionsetup[lstlisting]{singlelinecheck=false, labelfont={blue}, textfont={blue}}
\usepackage{caption}
\DeclareCaptionFont{white}{\color{white}}
\DeclareCaptionFormat{listing}{\colorbox{8}{\parbox{\textwidth}{\hspace{15pt}#1#2#3}}}
\captionsetup[lstlisting]{format=listing,labelfont=white,textfont=white, singlelinecheck=false, margin=0pt, font={bf,footnotesize}}

\usepackage[utf8x]{inputenc}
\usepackage{ngerman}




\title{Diskrete Minimalflächen}
\author{Patrick Dabbert, Stephan Hilb und Martin Rösner}
\date{\today}

\usepackage{beamerthemesplit}


\begin{document}

\begin{frame}
	\titlepage
\end{frame}

\begin{frame}
	\frametitle{Inhaltsverzeichnis}
	\tableofcontents[currentsection]
\end{frame}

\section{Allgemeines}

\subsection{Die Aufgabe und unsere Interpretation}

\begin{frame}
	\frametitle{Die Aufgabenstellung}
	\begin{quote}
		„Beschreiben Sie eine triangulierte Fläche in eine gegeben gekrümmte Raumkurve ein und versuchen Sie, eine Approximation an eine Minimalfläche zu berechnen.“
	\end{quote}
\end{frame}

\begin{frame}
	\frametitle{Unsere Überlegungen dazu}
	\begin{itemize}
		\item
			Kurve in parametrisierter Form: $\gamma(t)$
		\item
			Zur Minimierung ist ein Dreiecksgitter nötig
			
			$\quad \implies$ Triangulierung erforderlich
		\item
			Einfaches Ausgabeformat für die Visualisierung
	\end{itemize}
\end{frame}

\subsection{Die Datenstrukturen}

\begin{frame}
	\frametitle{Vertex}
	\lstinputlisting{vertex.cpp}
\end{frame}

\begin{frame}
	\frametitle{Edge}
	\lstinputlisting{edge.cpp}
\end{frame}

\begin{frame}
	\frametitle{Triangle}
	\lstinputlisting{triangle.cpp}
\end{frame}

\subsection{Das Ausgabeformat für die Visualisierung}

\begin{frame}[fragile]
	\frametitle{Das Wavefront OBJ-Format}
	\begin{itemize}
		\item
			Spezifikation von Punkten im Raum durch Zeilen der Form
			\begin{verbatim}
v [x-Koordinate] [y-Koordinate] [z-Koordinate]
			\end{verbatim}
			Implizite Indizierung der Punkte nach ihrer Reihenfolge.
		\item
			Verbindungslinie durch $n$ Punkte:
			\begin{verbatim}
l [ID Punkt 1] [ID Punkt 2] ... [ID Punkt n]
			\end{verbatim}
		\item
			Dreiecksfläche, definiert durch drei Punkte:
			\begin{verbatim}
f [ID Punkt 1] [ID Punkt 2] [ID Punkt 3]
			\end{verbatim}
	\end{itemize}
\end{frame}

\begin{frame}[fragile]
	\frametitle{Beispiel}
	\begin{verbatim}
v 0 0 0
v 1 0 0
v 0 1 0
v 0 0 1
l 1 4
f 1 2 3
	\end{verbatim}
\end{frame}


\section{Triangulierung}

\subsection{Vorüberlegungen}

\begin{frame}
	\frametitle{Manuelle Anfangstriangulierung}
	Definiere „per Hand“ ein Dreiecksnetz.

	Leider:
	\begin{itemize}
		\item
			Mühsam aufzustellen (für jede Kurve!)
		\item
			Erlaubt nur bedingt Annahmen über die Orientierung und Struktur des Netzes für die Verfeinerung.
		\item
			Ein Parser wäre nötig zum Einlesen der Definition (oder viel Quellcode tippen)
	\end{itemize}
	$\implies$ Idee verworfen
\end{frame}

\begin{frame}
	\frametitle{Iterative Verfeinerung}
	Beginne mit einem Dreieck mit Eckpunkten $\gamma(0), \gamma(\frac 13), \gamma(\frac 23)$, verfeinere dann iterativ.
	\begin{itemize}
		\item
			Eine geschlossene Kurve wird vorrausgesetzt
		\item
			Einfach nachzuvollziehen
		\item
			Es kann auch schief laufen (Beispiele später)
	\end{itemize}
\end{frame}

\subsection{Iterative Verfeinerung}


\section{Minimierung}

\subsection{Das lokale Minimierungsproblem}

\subsection{Das Gradientenverfahren}

\subsection{Implementierung}


\section{Ergebnisse}

\subsection{Schöne Minimalflächen}

\subsection{Weniger schöne Flächen und ihre Ursachen}

\subsection{Weiterführende Überlegungen}





\end{document}
