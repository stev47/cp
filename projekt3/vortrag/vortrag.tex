\documentclass{beamer}


\usepackage{color}
\usepackage{listings}
\usepackage{courier}
\lstset{
basicstyle=\tiny\ttfamily, % Standardschrift
numbers=left, % Ort der Zeilennummern
tabsize=4, % Groesse von Tabs
}
\lstloadlanguages{C++}
%\DeclareCaptionFont{blue}{\color{blue}}
 
%\captionsetup[lstlisting]{singlelinecheck=false, labelfont={blue}, textfont={blue}}
\usepackage{caption}
\DeclareCaptionFont{white}{\color{white}}
\DeclareCaptionFormat{listing}{\colorbox{8}{\parbox{\textwidth}{\hspace{15pt}#1#2#3}}}
\captionsetup[lstlisting]{format=listing,labelfont=white,textfont=white, singlelinecheck=false, margin=0pt, font={bf,footnotesize}}

\usepackage[utf8x]{inputenc}
\usepackage{ngerman}
\usepackage{graphicx}

\usepackage{longtable}


\title{Kleine Einfache Gruppen}
\author{Patrick Dabbert, Stephan Hilb und Martin R\"osner}
\date{\today}

\usepackage{beamerthemesplit}


\begin{document}

\begin{frame}
	\titlepage
\end{frame}

%\begin{frame}
%	\frametitle{Inhaltsverzeichnis}
%	\tableofcontents
%\end{frame}

\section{Aufgabe 1}
\subsection{Aufgabenstellung}

\begin{frame}
	\frametitle{Die Aufgabenstellung}
	\begin{quote}
	\textbf{Aufgabe 1:}\\
	Fertigen Sie eine Liste der endlichen einfachen nicht-abelschen Gruppen bis Ordnung 10.000 an.
	Sie dürfen hierbei verwenden, dass diese in folgender Liste enthalten sind: $A_n, PSL(2,q), PSL(3,3),
	PSU(3,3)$ und $M_{11}$. Verwenden sie, dass die Ordnung eines endlichen Körpers eine Primzahlpotenz ist.
	Welche der von ihnen gefundenen Gruppen sind minimal einfach?
		 
	\end{quote}
\end{frame}

\subsection{Definitionen}


\begin{frame}
	\frametitle{Definition: Einfache Gruppen}
	\begin{quote}
	 Eine Gruppe heißt einfach wenn sie nur 1 und sich selbst als Normalteiler besitzt.
		 
	\end{quote}
\end{frame}
\begin{frame}
	\frametitle{Definition Minimal Einfache Gruppen}
	\begin{quote}
	 Eine einfache Gruppe G heißt minimal einfach, wenn alle echten Untergruppen von G auflösbar sind.
	\end{quote}
\end{frame}

\subsection{Lösungsschritte}

\begin{frame}
	\frametitle{Lösungsschritte}
	\begin{quote}
	 \begin{enumerate}[1.]
	  \item gegebene Gruppen in Liste eintragen
	  \item Überprüfen ob Gruppen Einfach, Abelsch und Ordnung kleiner 10000.
	  \item Gruppen auf Isomorphie prüfen
	  \item Untergruppen auf Auflösbarkeit prüfen (hierbei reicht es die die Konjugationsklassen der Untergruppen
	  zu betrachten) $\Rightarrow$ Minimal einfach
	 \end{enumerate}

		 
	\end{quote}
\end{frame}
\subsection{Implementierung}

\begin{frame}
	\frametitle{Implementierung in GAP (Liste01.g)}
	 \lstinputlisting{liste01.g}
		 
\end{frame}

\begin{frame}
	\frametitle{Implementierung in GAP (Liste01.g)}
	 \lstinputlisting{liste01.1.g}
		 
\end{frame}

\begin{frame}
	\frametitle{Implementierung in GAP (gapfile01.g)}
	 \lstinputlisting{gapfile01.g}
		 
\end{frame}
\subsection{Ergebnisse}

\begin{frame}
	\frametitle{Ergebnis}
	 
\begin{tabular}{ll}
einfach & 	[ ''A7'', ''A5'', ''PSL(3,2)'', ''PSL(2,8)'',\\
&		 ''A6'', ''PSL(2,11)'', ''PSL(2,13)'', \\
&		 ''PSL(2,16)'',''PSL(2,17)'', ''PSL(2,19)'', \\
&		 ''PSL(2,23)'',''PSL(2,25)'', ''PSL(2,27)'', \\
&		''PSL(3,3)'', ''PSU(3,3)'',''M11'' ]\\
minimal einfach & [ ''A5'', ''PSL(3,2)'', ''PSL(2,8)'', \\ 
&		''PSL(2,13)'', ''PSL(2,17)'',''PSL(2,23)'', \\
&		''PSL(2,27)'', ''PSL(3,3)'' ]
	 \end{tabular}\\
	 
	 \textbf{Anmerkung}: PSL(3,2)$\cong$ PSL(2,7)
		 
\end{frame}



\section{Aufgabe 2}
\subsection{Aufgabenstellung}

\begin{frame}
	\frametitle{Die Aufgabenstellung}
	\begin{quote}
	\textbf{Aufgabe 2:}\\
	Zeigen Sie, dass $A_8$ und $ PSL(3,4)$ nicht isomorph sind.
		 
	\end{quote}
\end{frame}
\subsection{Lösungsschritte}

\begin{frame}
	\frametitle{Lösungsschritte}
	\begin{quote}
	 Überprüfe für $A_8$ und $PSL(3,4)$ für welche Ordnungen Elemente vorkommen.
		 
	\end{quote}
\end{frame}
\subsection{Implementierung}

\begin{frame}
	\frametitle{Implementierung in GAP (gapfile02.g)}
	\lstinputlisting{gapfile02.g}
\end{frame}
\subsection{Ergebnisse}

\begin{frame}
	\frametitle{Ergebnis}
	\begin{quote}
	  \textbf{Ausgabe}:
	  \lstinputlisting{ausgabe2.txt}
	  Es gibt also in $A_8$ Elemente von Ordnung 6 und 15, aber nicht in PSL(3,4). 
	  $\Rightarrow  A_8 \ncong PSL(3,4)$
		  
	\end{quote}
\end{frame}


\section{Aufgabe 3}
\subsection{Aufgabenstellung}

\begin{frame}
	\frametitle{Die Aufgabenstellung}
	\begin{quote}
	\textbf{Aufgabe 3}:\\
	(In Zusammenarbeit mit Partitionen-Gruppe)\\
	Bestimmen Sie, welche der von ihnen bestimmten Gruppen eine nicht-triviale Partition von Untergruppen besitzt.

	\end{quote}
\end{frame}
\subsection{Lösungsschritte}

\begin{frame}
	\frametitle{Lösungsschritte}
	\begin{quote}
	 \begin{enumerate}
	  \item Benutze erste Liste aus Aufgabe 1
	  \item Überprüfe für jede Gruppe ob nicht Triviale Partitionen vorliegen
	  \item Wenn ja, speichere die Gruppe in eine Liste
	 \end{enumerate}

		 
	\end{quote}
\end{frame}
\subsection{Implementierung}

\begin{frame}
	\frametitle{Implementierung in GAP (gapfile03.g)}
	\begin{quote}
	 \lstinputlisting{gapfile03.g}

		 
	\end{quote}
\end{frame}
\subsection{Ergebnisse}

\begin{frame}
	\frametitle{Ergebnis}
	\begin{quote}
	 Programm rechnet noch, Zwischenergebnisse bis PSL(3,3):\\
	 Mit nicht trivialen Partitionen: A5, PSL(3,2), PSL(2,8), A6 PSL(2,11), PSL(2,13), PSL(2,16), PSL(2,17), PSL(2,19), PSL(2,23), PSL(2,25), PSL(2,27)  \\
	 ohne: A7\\
	 noch nicht getestet: PSL(3,3), PSU(3,3), M11
	 
	 
		 
	\end{quote}
\end{frame}

\section{Aufgabe 4}
\subsection{Aufgabenstellung}

\begin{frame}
	\frametitle{Die Aufgabenstellung}
	\begin{quote}
	\textbf{Aufgabe 4:}\\
	(In Zusammenarbeit mit Sylowgruppen-Gruppe) Bestimmen Sie, welche der von Ihnen gefundenen Gruppen N-Gruppen sind. Geben Sie außerdem für 
	$ P \in Syl_2(G) $ und $ P \in Syl_3(G) $ die Isomorphietypen von P, Z(P), $ C_G(P) $ und $ N_G(P) $ an, wobei G eine Gruppe aus ihrer Liste ist.
	\end{quote}
\end{frame}
\subsection{Lösungsschritte}

\begin{frame}
	\frametitle{Lösungsschritte}
	\begin{enumerate}
	 \item Benutze erste Liste aus Teil eins.
	 \item Überprüfe die Gruppen darauf ob sie N Gruppen sind
	 \item wenn ja: Gib Isomorphietyp , Z(P), $ C_G(P) $ und $ N_G(P) $ an. 		 
	\end{enumerate}
\end{frame}
\subsection{Implementierung}

\begin{frame}
	\frametitle{Implementierung in GAP (gapfile04.g)}
	\begin{quote}
	 \lstinputlisting{gapfile04.g}
		 
	\end{quote}
\end{frame}
\subsection{Ergebnisse}

\begin{frame}
	\frametitle{Ergebnis}
\begin{quote}
 Alle getesteten Gruppen sind N-Gruppen. (Genaueres siehe Tabelle)
\end{quote}
		 
\section{Zusammenfassung}
\end{frame}
\begin{frame}
\begin{tabular}{ll}
einfach & 	[ ''A7'', ''A5'', ''PSL(3,2)'', ''PSL(2,8)'',\\
&		 ''A6'', ''PSL(2,11)'', ''PSL(2,13)'', \\
&		 ''PSL(2,16)'',''PSL(2,17)'', ''PSL(2,19)'', \\
&		 ''PSL(2,23)'',''PSL(2,25)'', ''PSL(2,27)'', \\
&		''PSL(3,3)'', ''PSU(3,3)'',''M11'' ]\\
minimal einfach & [ ''A5'', ''PSL(3,2)'', ''PSL(2,8)'', \\ 
&		''PSL(2,13)'', ''PSL(2,17)'',''PSL(2,23)'', \\
&		''PSL(2,27)'', ''PSL(3,3)'' ]\\
N-Gruppen & 	[ ''A7'', ''A5'', ''PSL(3,2)'', ''PSL(2,8)'',\\
&		 ''A6'', ''PSL(2,11)'', ''PSL(2,13)'', \\
&		 ''PSL(2,16)'',''PSL(2,17)'', ''PSL(2,19)'', \\
&		 ''PSL(2,23)'',''PSL(2,25)'', ''PSL(2,27)'', \\
&		''PSL(3,3)'', ''PSU(3,3)'',''M11'' ]\\

	 \end{tabular}\\
\end{frame}

\end{document}
